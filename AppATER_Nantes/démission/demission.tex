%%%%%%%%%%%%%%%%%%%%%%%%%%%%%%%%%%%%%%%%%
% Professional Formal Letter
% LaTeX Template
% Version 1.0 (28/12/13)
%
% This template has been downloaded from:
% http://www.LaTeXTemplates.com
%
% Original author:
% Brian Moses (http://www.ms.uky.edu/~math/Resources/Templates/LaTeX/)
% with extensive modifications by Vel (vel@latextemplates.com)
%
% License:
% CC BY-NC-SA 3.0 (http://creativecommons.org/licenses/by-nc-sa/3.0/)
%
%%%%%%%%%%%%%%%%%%%%%%%%%%%%%%%%%%%%%%%%%

%----------------------------------------------------------------------------------------
%	PACKAGES AND OTHER DOCUMENT CONFIGURATIONS
%----------------------------------------------------------------------------------------

\documentclass[11pt,a4paper]{letter} % Specify the font size (10pt, 11pt and 12pt) and paper size (letterpaper, a4paper, etc)

\usepackage{graphicx} % Required for including pictures
\usepackage{microtype} % Improves typography
\usepackage{gfsdidot} % Use the GFS Didot font: http://www.tug.dk/FontCatalogue/gfsdidot/
\usepackage[T1]{fontenc} % Required for accented characters
\usepackage{wasysym}
\usepackage{marvosym}
\usepackage{ifsym}
%\usepackage[misc,geometry]{ifsym}
\usepackage[utf8]{inputenc}
\usepackage{lmodern} % load a font with all the characters


% Create a new command for the horizontal rule in the document which allows thickness specification
\makeatletter
\def\vhrulefill#1{\leavevmode\leaders\hrule\@height#1\hfill \kern\z@}
\makeatother

%----------------------------------------------------------------------------------------
%	DOCUMENT MARGINS
%----------------------------------------------------------------------------------------

\textwidth 6.75in
\textheight 9.5in
\oddsidemargin -.25in
\evensidemargin -.25in
\topmargin -1.5in
\longindentation 0.50\textwidth
\parindent 0.3in

%----------------------------------------------------------------------------------------
%	SENDER INFORMATION
%----------------------------------------------------------------------------------------

\def\Who{Alejandro REYES-AMARO} % Your name
\def\What{, Doctorant} % Your title
%\def\Where{Laboratoire d'Informatique de Nantes Atlantique} % Your department/institution
\def\Address{\textifsymbol{18} 3 rue Docteur Zamenhof 44200 Nantes, Fance} % Your address
\def\CityZip{\hbox{} \hspace{4pt} 44200 Nantes, Fance} % Your city, zip code, country, etc
\def\Email{\Letter \hspace{0.1cm} alejandro.reyes@univ-nantes.fr} % Your email address
\def\TEL{\phone \hspace{0.1cm} (+33) 251-840861} % Your phone number
\def\MOBILE{\Mobilefone \hspace{0.1cm} (+33) 666-372524}
\def\URL{} % Your URL

%----------------------------------------------------------------------------------------
%	HEADER AND FROM ADDRESS STRUCTURE
%----------------------------------------------------------------------------------------

\address{
\includegraphics[width=1in]{logo.jpg} % Include the logo of your institution
\hspace{5.1in} % Position of the institution logo, increase to move left, decrease to move right
\vskip -1.07in~\\ % Position of the text in relation to the institution logo, increase to move down, decrease to move up
\Large\hspace{1.5in}{\sc Universit\'{e}} \hfill ~\\[0.05in] % First line of institution name, adjust hspace if your logo is wide
\hspace{1.5in}{\sc de Nantes} \hfill \normalsize % Second line of institution name, adjust hspace if your logo is wide
\makebox[0ex][r]{\bf \Who \What }\hspace{0.08in} % Print your name and title with a little whitespace to the right
~\\[-0.11in] % Reduce the whitespace above the horizontal rule
\hspace{1.5in}\vhrulefill{1pt} \\ % Horizontal rule, adjust hspace if your logo is wide and \vhrulefill for the thickness of the rule
\hspace{\fill}\parbox[t]{4in}{ % Create a box for your details underneath the horizontal rule on the right
\footnotesize % Use a smaller font size for the details
%\Who \\ \em % Your name, all text after this will be italicized
\em
%\Where\\ % Your department
\Address\\ % Your address
%\CityZip\\ % Your city and zip code
\TEL\\ % Your phone number
\MOBILE\\
\Email\\ % Your email address
%\URL % Your URL
}
\hspace{-1.4in} % Horizontal position of this block, increase to move left, decrease to move right
\vspace{-1in} % Move the letter content up for a more compact look
}

%----------------------------------------------------------------------------------------
%	TO ADDRESS STRUCTURE
%----------------------------------------------------------------------------------------
%\longindentation
\def\opening#1{\thispagestyle{empty}
{\centering\fromaddress \vspace{0.6in} \\ % Print the header and from address here, add whitespace to move date down
\hspace{6cm}13 Juin 2016\hspace*{\fill}\par} % Print today's date, remove \today to not display it
{\raggedright \toname \\ \toaddress \par} % Print the to name and address
\vspace{0.4in} % White space after the to address
\noindent #1 % Print the opening line
% Uncomment the 4 lines below to print a footnote with custom text
%\def\thefootnote{}
%\def\footnoterule{\hrule}
%\footnotetext{\hspace*{\fill}{\footnotesize\em Footnote text}}
%\def\thefootnote{\arabic{footnote}}
}

%----------------------------------------------------------------------------------------
%	SIGNATURE STRUCTURE
%----------------------------------------------------------------------------------------

\signature{{\bf \Who} } %\What} % The signature is a combination of your name and title

\long\def\closing#1{
\vspace{0.1in} % Some whitespace after the letter content and before the signature
\noindent % Stop paragraph indentation
\hspace*{\longindentation} % Move the signature right
\parbox{\indentedwidth}{\raggedright
#1 % Print the signature text
%\vskip 0.1in % Whitespace between the signature text and your name
\includegraphics[height=4.5\baselineskip]{pictures/sign}\\
\fromsig}} % Print your name and title

%----------------------------------------------------------------------------------------

\begin{document}

%----------------------------------------------------------------------------------------
%	TO ADDRESS
%----------------------------------------------------------------------------------------

\begin{letter}
{{\bf Université de Nantes}\\
Direction des Ressources Humaines\\
44035 Nantes, France\\
}

%----------------------------------------------------------------------------------------
%	LETTER CONTENT
%----------------------------------------------------------------------------------------

\opening{Madame, Monsieur,}

Je vous annonce par la présente lettre que je prends la décision de démissioner de mon poste de \textbf{Doctorant} au sein du \textit{Laboratoire d'Informatique de Nantes Atlantique de l'Université de Nantes}.

J'accepte d'occuper mon post actuel jusqu'au 01 Septembre 2016, date ou je commence à occuper mon post d'\textbf{Attaché Temporaire d'Enseignement et de Recherche} à \textit{l'UFR de Sciences et Techniques (Université de Nantes)}. 

Merci de bien vouloir accepter ma décision et je vous pris d'agréer l'expression de mes salutations distinguées.

%Je vous écris pour postuler en tant qu'Attaché Temporaire d'Enseignement et de Recherche ATER au Département d'Informatique de l'Université d'Angers. Actuellement, je suis en dernière année de doctorat au Laboratoire d'Informatique de Nantes Atlantique (LINA) à l'Université de Nantes, sous la supervision d'Éric MONFROY (directeur de thèse) et Florian RICHOUX (encadrant). Grâce à ce poste, je suis également moniteur d'initiation à l'enseignement supérieur auprès des L1. 

%Ma thèse porte sur des nouvelles méthodes en parallèle pour résoudre des problèmes d'optimisation combinatoire, et je souhaite la soutenir fin décembre 2016. Une des raisons pour laquelle je postule à ce poste est d'avoir un financement afin de pouvoir finir ma thèse, vu que mon contrat se finit en octobre 2016. Mon expérience comme enseignant dans plusieurs domaines depuis 2003 (MPI, POO, analyse numérique, etc.), et ma maîtrise de plusieurs langages de programmation (C/C++/C\#, Java/Javascript), feront que la préparation des cours et la fin de ma thèse seront compatibles. J'ai su à travers de M. Frédéric LARDEUX que la programmation parallèle est enseigné à l'Université d'Angers mais que peu d'enseignants veulent le faire. Dans ce cas, vu que c'est un des cœurs de ma thèse, je pense pouvoir vous apporter avec mon expérience dans ce domaine. 

%Pendant mon doctorat, j'ai pu acquérir beaucoup de connaissances en programmation par contraintes, en parallélisme et en méta-heuristique. J’ai développé un langage permettant la résolution de problèmes d’optimisation combinatoire, avec une nouvelle approche mettant particulièrement l’accent sur l’obtention d’un cadre extensible en facilitant l’incorporation de nouvelles fonctionnalités et versatile en comptant avec une vaste gamme de modules réutilisables.

%Je suis convaincu qu'avec l'expérience que j'ai acquise au cours de mes études à Nantes, je peux m'impliquer positivement dans votre groupe avec de l'intérêt et de la créativité. Je suis une personne très responsable, très travailleur, capable d’apprendre rapidement et très flexible, m'adaptant aux besoinsde la situation.

%Vous trouverez ci-joint tous les documents constituant mon dossier de candidature.
%Dans l'attente de votre réponse, je vous prie de croire, Madame, Monsieur, à l'assurance de toute ma considération.
%Dans l’attente d'une réponse positive de votre part, je vous prie d’agréer, Madame, Monsieur, l’expression de mes sincères salutations.
%Dans l’attente de vous rencontrer, je vous prie de croire, Madame, Monsieur, à l’expression de mes sincères salutations.
%Je me tiens \`{a} votre entière disposition pour tous renseignements complémentaires. Je vous prie, Madame, Monsieur, d'agréer l'expression de mes respectueuses salutations.

\closing{Bien cordialement,}

%----------------------------------------------------------------------------------------

\end{letter}
\end{document}
