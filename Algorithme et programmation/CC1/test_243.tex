% Exam Template for UMTYMP and Math Department courses
%
% Using Philip Hirschhorn's exam.cls: http://www-math.mit.edu/~psh/#ExamCls
%
% run pdflatex on a finished exam at least three times to do the grading table on front page.
%
%%%%%%%%%%%%%%%%%%%%%%%%%%%%%%%%%%%%%%%%%%%%%%%%%%%%%%%%%%%%%%%%%%%%%%%%%%%%%%%%%%%%%%%%%%%%%%

% These lines can probably stay unchanged, although you can remove the last
% two packages if you're not making pictures with tikz.
\documentclass[11pt]{exam}
\RequirePackage{amssymb, amsfonts, amsmath, latexsym, verbatim, xspace, setspace}
\RequirePackage{tikz, pgflibraryplotmarks}

% By default LaTeX uses large margins.  This doesn't work well on exams; problems
% end up in the "middle" of the page, reducing the amount of space for students
% to work on them.
\usepackage[margin=1in]{geometry}

\usetikzlibrary{calc}
\usepackage{graphicx}

% Here's where you edit the Class, Exam, Date, etc.
\newcommand{\class}{Algorithmique et programmation}
\newcommand{\term}{X2I0010, groupe 243}
\newcommand{\examnum}{CC 1}
\newcommand{\examdate}{30/01/2017}
\newcommand{\timelimit}{30 Minutes}

\usepackage{fancyvrb}
\fvset{frame=single,framesep=1mm,fontfamily=courier,fontsize=\normalsize,numbers=left,framerule=.3mm,numbersep=1mm,commandchars=\\\{\}}
\definecolor{dgreen}{rgb}{0.0, 0.5, 0.0}
\definecolor{ballblue}{rgb}{0.13, 0.67, 0.8}

\usepackage[absolute,overlay]{textpos}
\usepackage{caption}

% For an exam, single spacing is most appropriate
\singlespacing
% \onehalfspacing
% \doublespacing

% For an exam, we generally want to turn off paragraph indentation
\parindent 0ex

\begin{document} 

\begin{tikzpicture}[overlay, remember picture]
\node[anchor=north west, %anchor is upper left corner of the graphic
      xshift=2.5cm, %shifting around
      yshift=-0.8cm] 
     at (current page.north west) %left upper corner of the page
     {\includegraphics[width=16.4cm]{UN_logo.png}}; 
\end{tikzpicture}

% These commands set up the running header on the top of the exam pages
\pagestyle{head}
\firstpageheader{}{}{}
\runningheader{\class}{\examnum\ - Page \thepage\ of \numpages}{\examdate}
\runningheadrule

\begin{flushright}
\begin{tabular}{p{2.8in} r l}
\textbf{\class} && \\%\makebox[2in]{\hrulefill}\\
\textbf{\term} &&\\
\textbf{\examnum} &&\\
\textbf{\examdate} &&\\
\textbf{Dur\'{e}e: \timelimit} & \textbf{Nom, Pr\'enom:} & \makebox[2in]{\hrulefill}
\end{tabular}\\
\end{flushright}
\rule[1ex]{\textwidth}{.1pt}

\noindent\fbox{%
\parbox{0.98\textwidth}{%
\textbf{Pr\'eambule} : Aucun document autoris\'e. Calculatrices et t\'el\'ephones portables interdits. %Les exercices ne sont pas class\'es par difficult\'e croissante. Nombre de pages : \numpages.
}%
}
\vspace{12pt}

%\begin{minipage}[t]{3.7in}
%\vspace{0pt}
%\begin{itemize}
%
%\item \textbf{If you use a ``fundamental theorem'' you must indicate this} and explain
%why the theorem may be applied.
%
%\item \textbf{Organize your work}, in a reasonably neat and coherent way, in
%the space provided. Work scattered all over the page without a clear ordering will 
%receive very little credit.  
%
%\item \textbf{Mysterious or unsupported answers will not receive full
%credit}.  A correct answer, unsupported by calculations, explanation,
%or algebraic work will receive no credit; an incorrect answer supported
%by substantially correct calculations and explanations might still receive
%partial credit.
%
%
%\item If you need more space, use the back of the pages; clearly indicate when you have done this.
%\end{itemize}
%
%Do not write in the table to the right.
%\end{minipage}
%\hfill
%\begin{minipage}[t]{2.3in}
%\vspace{0pt}
%%\cellwidth{3em}
%\gradetablestretch{2}
%\vqword{Problem}
%\addpoints % required here by exam.cls, even though questions haven't started yet.	
%\gradetable[v]%[pages]  % Use [pages] to have grading table by page instead of question
%
%\end{minipage}
%	\newpage % End of cover page

%%%%%%%%%%%%%%%%%%%%%%%%%%%%%%%%%%%%%%%%%%%%%%%%%%%%%%%%%%%%%%%%%%%%%%%%%%%%%%%%%%%%%
%
% See http://www-math.mit.edu/~psh/#ExamCls for full documentation, but the questions
% below give an idea of how to write questions [with parts] and have the points
% tracked automatically on the cover page.
%
%
%%%%%%%%%%%%%%%%%%%%%%%%%%%%%%%%%%%%%%%%%%%%%%%%%%%%%%%%%%%%%%%%%%%%%%%%%%%%%%%%%%%%%

\begin{questions}

% Basic question
%\addpoints
\question \'Ecrire un algorithme qui demande de saisir un premier mot puis un second et retourne le nombre minimal de rotations \`a effectuer sur le premier mot pour obtenir le second mot. L'algorithme a trois sorties possibles :
\begin{enumerate}
\item ``impossible'', 
\item ``$x$ gauche'',
\item ``$y$ droite''
\end{enumerate}

o\`u $x$ (resp. $y$) est le nombre de rotations vers la gauche (resp. droite) \`a effectuer.

\underline{Exemple :} 

\begin{Verbatim}
\textcolor{blue}{\bf ~$} prendre
\textcolor{blue}{\bf ~$} reprend
\textcolor{blue}{\bf ~$} 2 droite
\end{Verbatim}



\end{questions}
\end{document}