% Exam Template for UMTYMP and Math Department courses
%
% Using Philip Hirschhorn's exam.cls: http://www-math.mit.edu/~psh/#ExamCls
%
% run pdflatex on a finished exam at least three times to do the grading table on front page.
%
%%%%%%%%%%%%%%%%%%%%%%%%%%%%%%%%%%%%%%%%%%%%%%%%%%%%%%%%%%%%%%%%%%%%%%%%%%%%%%%%%%%%%%%%%%%%%%

% These lines can probably stay unchanged, although you can remove the last
% two packages if you're not making pictures with tikz.
\documentclass[11pt]{exam}
\RequirePackage{amssymb, amsfonts, amsmath, latexsym, verbatim, xspace, setspace}
\RequirePackage{tikz, pgflibraryplotmarks}

% By default LaTeX uses large margins.  This doesn't work well on exams; problems
% end up in the "middle" of the page, reducing the amount of space for students
% to work on them.
\usepackage[margin=1in]{geometry}

\usetikzlibrary{calc}
\usepackage{graphicx}

\usepackage{enumitem}

% Here's where you edit the Class, Exam, Date, etc.
\newcommand{\class}{Algorithmique et programmation}
\newcommand{\term}{X2I0010, groupe 243}
\newcommand{\examnum}{CC 4}
\newcommand{\examdate}{24/04/2017}
\newcommand{\timelimit}{30 Minutes}

\usepackage{fancyvrb}
\fvset{frame=single,framesep=1mm,fontfamily=courier,fontsize=\normalsize,numbers=left,framerule=.3mm,numbersep=1mm,commandchars=\\\{\}}
\definecolor{dgreen}{rgb}{0.0, 0.5, 0.0}
\definecolor{ballblue}{rgb}{0.13, 0.67, 0.8}

\usepackage[absolute,overlay]{textpos}
\usepackage{caption}

% For an exam, single spacing is most appropriate
\singlespacing
% \onehalfspacing
% \doublespacing

% For an exam, we generally want to turn off paragraph indentation
\parindent 0ex

\begin{document} 

\begin{tikzpicture}[overlay, remember picture]
\node[anchor=north west, %anchor is upper left corner of the graphic
      xshift=2.5cm, %shifting around
      yshift=-0.8cm] 
     at (current page.north west) %left upper corner of the page
     {\includegraphics[width=16.4cm]{UN_logo.png}}; 
\end{tikzpicture}

% These commands set up the running header on the top of the exam pages
\pagestyle{head}
\firstpageheader{}{}{}
\runningheader{\class}{\examnum\ - Page \thepage\ of \numpages}{\examdate}
\runningheadrule

\begin{flushright}
\begin{tabular}{p{2.8in} r l}
\textbf{\class} && \\%\makebox[2in]{\hrulefill}\\
\textbf{\term} &&\\
\textbf{\examnum} &&\\
\textbf{\examdate} &&\\
\textbf{Dur\'{e}e: \timelimit} & \textbf{Nom, Pr\'enom:} & \makebox[2in]{\hrulefill}
\end{tabular}\\
\end{flushright}
\rule[1ex]{\textwidth}{.1pt}

\noindent\fbox{%
\parbox{0.98\textwidth}{%
%\textbf{Pr\'eambule} : Aucun document autoris\'e. Calculatrices et t\'el\'ephones portables interdits. Les exercices ne sont pas class\'es par difficult\'e croissante. Nombre de pages : \numpages.
\begin{tabular}{ll}
\textbf{Pr\'eambule} : & Aucun document autoris\'e. \\
& Calculatrices et t\'el\'ephones portables interdits. \\
& Les exercices ne sont pas class\'es par difficult\'e croissante. \\
& Nombre de pages : \numpages.
\end{tabular}
}%
}
\vspace{12pt}

%\begin{minipage}[t]{3.7in}
%\vspace{0pt}
%\begin{itemize}
%
%\item \textbf{If you use a ``fundamental theorem'' you must indicate this} and explain
%why the theorem may be applied.
%
%\item \textbf{Organize your work}, in a reasonably neat and coherent way, in
%the space provided. Work scattered all over the page without a clear ordering will 
%receive very little credit.  
%
%\item \textbf{Mysterious or unsupported answers will not receive full
%credit}.  A correct answer, unsupported by calculations, explanation,
%or algebraic work will receive no credit; an incorrect answer supported
%by substantially correct calculations and explanations might still receive
%partial credit.
%
%
%\item If you need more space, use the back of the pages; clearly indicate when you have done this.
%\end{itemize}
%
%Do not write in the table to the right.
%\end{minipage}
%\hfill
%\begin{minipage}[t]{2.3in}
%\vspace{0pt}
%%\cellwidth{3em}
%\gradetablestretch{2}
%\vqword{Problem}
%\addpoints % required here by exam.cls, even though questions haven't started yet.	
%\gradetable[v]%[pages]  % Use [pages] to have grading table by page instead of question
%
%\end{minipage}
%	\newpage % End of cover page

%%%%%%%%%%%%%%%%%%%%%%%%%%%%%%%%%%%%%%%%%%%%%%%%%%%%%%%%%%%%%%%%%%%%%%%%%%%%%%%%%%%%%
%
% See http://www-math.mit.edu/~psh/#ExamCls for full documentation, but the questions
% below give an idea of how to write questions [with parts] and have the points
% tracked automatically on the cover page.
%
%
%%%%%%%%%%%%%%%%%%%%%%%%%%%%%%%%%%%%%%%%%%%%%%%%%%%%%%%%%%%%%%%%%%%%%%%%%%%%%%%%%%%%%

\begin{questions}

% Basic question
%\addpoints
\question Dans la discipline de l'analyse num\'erique, une matrice creuse est une matrice contenant beaucoup de z\'eros. Quand on veut manipuler ou stocker des matrices creuses en informatique, il est avantageux d'utiliser des algorithmes et des structures de donn\'ees qui prennent en compte la structure peu dense de la matrice \footnote{Wikip\'edia (https://fr.wikipedia.org/wiki/Matrice\_creuse)}. 

L'objectif de l'exercice c'est d'\'ecrire des fonctions pour pouvoir manipuler des matrices creuses, \`a l'aide des types \textit{enregistrements} \texttt{T\_ligne\_creuse} et \texttt{T\_matrice\_creuse} donn\'es ci-dessous. 
 

\begin{Verbatim}
\textbf{type} \textcolor{blue}{\bf T\_ligne\_creuse} = \textbf{enregistrement}
	\textbf{pointeur vers tableau d'entiers} indices
	\textbf{pointeur vers tableau d'entiers} valeurs
	\textbf{entier} nb  \textcolor{dgreen}{/* nombre d'\'el\'ements diff\'erents de z\'ero */}
\textbf{fin enregistrement} 
\end{Verbatim}

\begin{Verbatim}
\textbf{type} \textcolor{blue}{\bf T\_matrice\_creuse} = \textbf{enregistrement}
	\textbf{pointeur vers tableau de} \textcolor{blue}{\bf T\_ligne\_creuse} lignes
	\textbf{entier} n \textcolor{dgreen}{/* nombre de lignes */}
	\textbf{entier} m \textcolor{dgreen}{/* nombre de colonnes */}
\textbf{fin enregistrement} 
\end{Verbatim}

\textbf{Question 1 :} Sp\'ecifiez trois fonctions dont les r\^oles sont, respectivement de :
\begin{enumerate}[label=\alph*)]
\item V\'erifier si une matrice creuse a \'et\'e correctement d\'eclar\'ee (si le nombre d'\'el\'ements dans chaque ligne est compatible avec le nombre de colonnes de la matrice);
\item V\'erifier si deux matrices creuses peuvent \^etre additionn\'ees;
\item \'Etant donnée la fonction \texttt{element\_dans} ci-dessous, retourner la somme de deux matrices creuses dans un tableau d'entiers \`a deux dimensions.
\end{enumerate}

\newpage

\begin{Verbatim}
\textbf{fonction} element_dans(\textcolor{blue}{\bf T\_ligne\_creuse} ligne, \textbf{entier} colonne) : \textbf{entier}
\textbf{variables :}
	\textbf{entier} i, resultat
\textbf{debut}
    resultat <- 0
    i <- 1
    \textbf{tant que}( i <= colonne \textbf{et} i <= ligne.nb) \textbf{faire}
        \textbf{si} (\textbf{memoire}(ligne.indices[i]) = colonne) \textbf{alors}
            resultat <- \textbf{memoire}(ligne.valeurs[i]);
        \textbf{fin si}
    \textbf{fin tant que}
    \textbf{returner} resultat;
\textbf{fin}
\end{Verbatim}

\underline{Exemple :} La matrice creuse
\[
A_{4\times 5} = 
\left[ {\begin{array}{ccccc}
2 & 0 & 0 & 1 & 0\\
0 & 0 & 1 & 0 & 0\\
3 & 0 & 0 & 1 & 2\\
0 & 0 & 0 & 0 & 1
\end{array} } \right]
\]

\`a la repr\'esentation suivante :

\begin{tabular}{llcll}
$A \rightarrow$ & ligne 1 (2 \'el\'ements) & $\rightarrow$ & indices : $\{1,4\}$ & valeurs : $\{2,1\}$ \\
& ligne 2 (1 \'el\'ement) & $\rightarrow$ & indices : $\{3\}$ & valeurs : $\{1\}$ \\
& ligne 3 (3 \'el\'ements) & $\rightarrow$ & indices : $\{1,4,5\}$ & valeurs : $\{3,1,2\}$ \\
& ligne 4 (1 \'el\'ement) & $\rightarrow$ & indices : $\{5\}$ & valeurs : $\{1\}$
\end{tabular}

Alors,

\[
A_{4\times 5} + B_{4\times 5} =  
\left[ {\begin{array}{ccccc}
2 & 0 & 0 & 1 & 1\\
0 & 0 & 1 & 0 & 0\\
4 & 1 & 0 & 1 & 2\\
0 & 1 & 0 & 0 & 1
\end{array} } \right]
\]

o\`u :

\begin{tabular}{llcll}
$B \rightarrow$ & ligne 1 (1 \'el\'ement) & $\rightarrow$ & indices : $\{5\}$ & valeurs : $\{1\}$ \\
& ligne 2 (0 \'el\'ement) & $\rightarrow$ & indices : $\phi$ & valeurs : $\phi$ \\
& ligne 3 (2 \'el\'ements) & $\rightarrow$ & indices : $\{1,2\}$ & valeurs : $\{1,1\}$ \\
& ligne 4 (1 \'el\'ement) & $\rightarrow$ & indices : $\{2\}$ & valeurs : $\{1\}$
\end{tabular}
\end{questions}
\end{document}