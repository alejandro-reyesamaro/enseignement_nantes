%================================================================
% TP2-X4I0010.tex
% L2_X4I0010: Système d'exploitation
%====================================================================
\documentclass[CC,sansRappel,12pt]{tdtp-utf8}
\usepackage{graphicx}
\newcommand{\regle}[2]{$ #1 \rightarrow #2 $}
\newcommand{\tab}{\hspace{0.5cm}}
\newcommand{\ra}{\rightarrow}
\newcommand{\RA}{\Rightarrow}

\cursus{L2 Informatique}
\intitule{X4I0010 -- Système d'exploitation}
\date{}
\title{Contrôle de travaux pratiques -- Contrôle final}

\usepackage{fancyvrb}
\fvset{frame=single,framesep=1mm,fontfamily=courier,fontsize=\normalsize,framerule=.3mm,numbersep=1mm,commandchars=\\\{\}}
\definecolor{dgreen}{rgb}{0.0, 0.5, 0.0}
\definecolor{ballblue}{rgb}{0.13, 0.67, 0.8}
\newcommand{\totranslate}[1]{\textcolor{blue}{#1}}
\newcommand{\tocorrect}[1]{\textcolor{dgreen}{#1}}

\begin{document}

%====================================================================
\begin{feuille}

\hspace*{1mm} \hfill Épreuve de contrôle continu du 27 avril 2017 (groupe 402)\\
\hspace*{1mm} \hfill Tous documents autorisés\\[5mm]

\fbox{\parbox{0.9\textwidth}{
%Vous rédigerez un compte-rendu sur lequel vous
%indiquerez la réponse à chaque question, vos
%explications et commentaires (interprétation
%du résultat), et le cas échéant, la ou les commandes
%utilisées.
%
Déposez votre devoir sous la forme d'une seule archive compressée sur {\sc Madoc} dans l'espace prévu à cet effet (Dépôt des compte-rendus de TP CC $\rightarrow$ Groupe 402 : Contrôle continu de TP -- Contrôle final). Veillez bien à ce que votre archive comporte les noms du binôme de TP (ex : \texttt{dupont-durant\_TPCC\_Controle\_final.zip}).
Le contenu de l'archive sera constituée des fichiers sources commentés des 2 programmes.}}

\vspace{20pt}

Dans un parc naturel nous avons introduit une population de cerfs (deers) avec l'objectif d'encourager la chasse chez les habitants de la région. Dans le même temps nous souhaitons que la population de cerfs ne décroise pas rapidement. Pour cette raison nous disposons d'un inspecteur chargé de vérifier l'état de la population une fois par mois.

À partir de la fréquence de reproduction des cerfs et de divers facteurs (météorologique, alimentation, prédateur, etc.) il est établi des règles indiquant les mois de l'année où la chasse des cerfs est autorisée dans le parc.

L'objectif de cet exercice est de simuler l'évolution de la population de cerfs dans le parc naturel au fil des mois.

\begin{exercice}[Les processus]

\tocorrect{Dans un premier temps, nous souhaitons poublier les régles de chasse du parc et les periodes de réproduction des cerfs, pour chauqe mois. Pour cela, nous devons utiliser deus processus differents:}

\begin{itemize}
\item \tocorrect{Un processus \textbf{A} qui n'a pas le droit d'afficher le mot ``\textit{hunting}'', et}
\item \tocorrect{Un processus \textbf{B} qui n'a pas le droit d'afficher le mot ``\textit{reproduction}''.}
\end{itemize}

\textbf{Question 1: } \tocorrect{En utilisant des processus et des signaux, completez le programme \texttt{proc\_2017.cpp} (à télécharger depuis {\sc Madoc}) pour pouvoir publier les règles du parc (présents dans le fichier \texttt{np\_rules.txt}) comme le montre l'exemple suivent :}
%\underline{Exemple 1 :}
\vspace{10pt}

Contenu du fichier \texttt{np\_rules.txt} :
\begin{Verbatim}
000111111101 
101011111110 
\end{Verbatim}

Texte à afficher :

\begin{Verbatim}
1 -> Only hunting 
2 -> Nothing
3 -> Only hunting 
4 -> Only reproduction
5 -> Reproduction and hunting 
6 -> Reproduction and hunting 
7 -> Reproduction and hunting 
8 -> Reproduction and hunting 
9 -> Reproduction and hunting 
10 -> Reproduction and hunting 
11 -> Only hunting 
12 -> Only reproduction
\end{Verbatim}

%Écrire un programme C (ou C++) qui crée deux processus fils affichant toutes les secondes leur PID ainsi que la valeur d'un compteur incrémenté de un chaque seconde. De son coté, le processus père attendra dix secondes avant d'envoyer un signal qui devra ``tuer'' le 1er fils. Il attendra encore dix secondes pour faire la même chose à son second fils.
\end{exercice}

\begin{exercice}[Les threads et les sémaphores]

\tocorrect{Le programme \texttt{thr\_2017.cpp} (à télécharger depuis {\sc Madoc}) contient une simulation en utilisant la fonction \texttt{sleep(...)}, du comportement que nous souhaitons étudier dans le parc naturel. Chaque personnage (cerfs, chasseurs, inspecteur) sera représenté par un fil (\textit{thread}) d'exécution différent. Chaque un devra suivre strictement les règles suivantes :}

\begin{itemize}
\item \underline{Cerfs :}
\begin{enumerate}
\item Les cerfs \tocorrect{se reproduisent une seule fois pendant les premiers 10 jours de chaque mois indiqué dans les règles generales du parc (fichier \texttt{np\_rules.txt}), en respectant la loi suivante :} $$pop_{i+1} = 1.15\cdot pop_{i}$$
\item[] où $pop_k$ est la population des cerfs du mois $k$.
\item Les cerfs \tocorrect{peuvent se réproduire le mois $X$ si l'inspecteur a déjà publié les résultats de l'inspection du mois $X-1$}
\end{enumerate}
\item \underline{Chasseurs:}
\begin{enumerate}
\item Les chasseurs \tocorrect{ont le droit à chasser seulement un jour dans la deuxième quinzaine de chauqe mois indiqué dans les règles generales du parc (fichier \texttt{np\_rules.txt}).}
\item Les chasseurs \tocorrect{ont le droit à chasser un maximum de 25 animaux par mois.}
\item[] \textbf{(bonus) : } \tocorrect{Implementez un fonction de probabilité (uniforme, par exemple) pour faire varier ce nombre à chaque fois.}
\item Les chasseurs \tocorrect{peuvent chasser le mois $X$ si l'inspecteur a déjà publié les résultats de l'inspection du mois $X-1$.}
\end{enumerate}
\item \underline{Inspecteur:}
\begin{enumerate}
\item L'inspecteur \tocorrect{publie les résultats de l'inspection le dernier jour de chaque mois à 23h59, en affichant la population actuel des cerfs dans le parc naturel.}
\end{enumerate}
\end{itemize}

\textbf{Question 1 : } \tocorrect{En utilisant des \textit{threads} et des sémaphores, modifiez le programme \texttt{thr\_2017.cpp} (sans utiliser la fonction \texttt{sleep(...)}) pour obtenir le résultat qu'il affiche tout en respectant les lois décrites} \totranslate{anteriormente}.

\textbf{Question 2 (bonus) :} \tocorrect{Modifiez le contenu du fichier \texttt{np\_rules.txt} } \totranslate{con el objetivo de lograr que la población de ciervos llegue a nuestros días.}

%Nous voulons écrire un programme qui crée deux threads en plus du thread principal (3 threads en tout). Les threads créés devront accéder à une ressource commune constituée d'une variable entière \textbf{x} globale à l'application. Un des deux threads auxiliaires devra sans cesse incrémenter la variable \textbf{x} de un. L'autre thread devra sans cesse la décrémenter de un. De plus, chaque thread conservera le nombre de fois qu'il a modifié \textbf{x}. Ainsi, normalement le contenu de \textbf{x} devrait être identique à la différence du nombre d'incrémentations de \textbf{x} et du nombre de décrémentations de \textbf{x}. Le thread principal devra afficher la valeur de la variable \textbf{x} toutes les secondes. Il affichera aussi la différence entre le nombre d'opérations effectuées par le premier thread et celui du second thread (pour cette raison, ces deux compteurs seront aussi implémentés comme des variables globales partagées par les 3 threads).

%\begin{enumerate}
%\item Écrire un premier programme sans prendre de précaution particulière pour l'accès aux ressources partagées par les 3 threads. Normalement, l'exécution de ce programme devrait montrer rapidement une divergence entre le contenu variable \textbf{x} et la différence des comptages des opérations d'incrémentation et de décrémentation. 
%\item Écrire un second programme utilisant un sémaphore pour créer un ``mutex'' permettant de protéger l'accès concurrent aux ressources partagées. L'exécution de ce programme ne devrait pas montrer de divergence entre le contenu de \textbf{x} et la différence des comptages des opérations effectuées par les deux threads auxiliaires.
%\end{enumerate}

\end{exercice}

\end{feuille}

\end{document}

