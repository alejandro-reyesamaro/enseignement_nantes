%================================================================
% TP2-X4I0010.tex
% L2_X4I0010: Système d'exploitation
%====================================================================
\documentclass[CC,sansRappel,12pt]{tdtp-utf8}
\usepackage{graphicx}
\newcommand{\regle}[2]{$ #1 \rightarrow #2 $}
\newcommand{\tab}{\hspace{0.5cm}}
\newcommand{\ra}{\rightarrow}
\newcommand{\RA}{\Rightarrow}

\cursus{L2 Informatique}
\intitule{X4I0010 -- Système d'exploitation}
\date{}
\title{Contrôle de travaux pratiques -- Contrôle final}

\begin{document}

%====================================================================
\begin{feuille}

\hspace*{1mm} \hfill Épreuve de contrôle continu du 28 avril 2016 (groupe 402)\\
\hspace*{1mm} \hfill Tous documents autorisés\\[5mm]

\fbox{\parbox{0.9\textwidth}{
%Vous rédigerez un compte-rendu sur lequel vous
%indiquerez la réponse à chaque question, vos
%explications et commentaires (interprétation
%du résultat), et le cas échéant, la ou les commandes
%utilisées.
%
Déposez votre devoir sous la forme d'une seule archive
compressée sur Madoc dans l'espace
prévu à cet effet (Dépôt des compte-rendus de TP CC 
$\rightarrow$ Groupe 402 : Contrôle continu de TP --
Contrôle final). Veillez bien à ce que votre archive 
comporte les noms du binôme de TP (ex :
dupont-durant\_TPCC\_Controle\_final.zip).
Le contenu de l'archive sera constituée des fichiers sources
commentés des 3 programmes.
}}

\begin{exercice}[Les processus]

Écrire un programme C (ou C++) qui crée deux processus fils affichant
toutes les secondes leur PID ainsi que la valeur d'un compteur incrémenté
de un chaque seconde.
De son coté, le processus père attendra dix secondes avant d'envoyer
un signal qui devra ``tuer'' le 1er fils. Il attendra encore dix secondes
pour faire la même chose à son second fils.
\end{exercice}

\begin{exercice}[Les threads et les sémaphores]

Nous voulons écrire un programme qui crée deux threads
en plus du thread principal (3 threads en tout).
Les threads créés devront accéder à une ressource commune
constituée d'une variable entière \textbf{x} globale à l'application.
Un des deux threads auxiliaires devra sans cesse incrémenter la
variable \textbf{x} de un. L'autre thread devra sans cesse la décrémenter
de un.
De plus, chaque thread conservera le nombre de fois qu'il a modifié
\textbf{x}. Ainsi, normalement le contenu de \textbf{x}
devrait être identique à la différence du nombre
d'incrémentations de
\textbf{x} et du nombre de décrémentations de \textbf{x}.
Le thread principal devra afficher la valeur de la variable \textbf{x}
toutes les secondes. Il affichera aussi la différence entre le nombre
d'opérations effectuées par le premier thread et celui du second thread
(pour cette raison, ces deux compteurs seront aussi implémentés comme
des variables globales partagées par les 3 threads).

\begin{enumerate}
\item Écrire un premier programme sans prendre de précaution particulière
pour l'accès aux ressources partagées par les 3 threads. Normalement,
l'exécution de ce programme devrait montrer rapidement une divergence entre
le contenu variable \textbf{x} et la différence des comptages des opérations
d'incrémentation et de décrémentation.
\item Écrire un second programme utilisant un sémaphore pour créer un
``mutex'' permettant de
protéger l'accès concurrent aux ressources partagées. L'exécution de
ce programme ne devrait pas montrer de divergence entre le contnu
de \textbf{x} et
la différence des comptages des opérations effectuées par les deux threads
auxiliaires.
\end{enumerate}

\end{exercice}

\end{feuille}

\end{document}

