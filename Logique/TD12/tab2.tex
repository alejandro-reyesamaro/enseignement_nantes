\documentclass[a4paper,10pt]{article}
\usepackage[utf8]{inputenc}
\usepackage{tikz}
\usetikzlibrary{matrix,shapes,arrows,positioning,chains}

\addtolength{\topmargin}{-.875in}
\addtolength{\textheight}{1.5in}
\addtolength{\textwidth}{1.7in}
\addtolength{\oddsidemargin}{-.875in}
\addtolength{\evensidemargin}{-.875in}

\usepackage{amsfonts}
\usepackage{amssymb}
\usepackage{amsmath}
\usepackage[section]{placeins}

\usepackage{turnstile}
\newcommand{\ded}[2]{#1\hspace{3pt}$\sdtstile{}{}$\hspace{3pt}#2}
\newcommand{\Ded}[1]{$\sdtstile{}{}$\hspace{3pt}#1}
\newcommand{\imp}{\Rightarrow}
\newcommand{\spc}{.\hspace{5pt}}

\newcommand{\oneNODEone}{%
\begin{align*}
1\spc& \forall x_1\, \left[ep(x_1) \land \lnot hpe(x_1) \imp \exists y\, \left[dn(y) \land fl(y,x_1)\right]\right]\\
  & \equiv \forall x\, \exists y\, \left[\left(\lnot ep(x_1) \lor hpe(x_1) \lor dn(y)\right) \land \left(\lnot ep(x_1) \lor hpe(x_1) \lor fl(y,x_1)\right)\right]\\
2\spc& \exists x\, \left[sc(x)\land ep(x) \land \forall y\, \left[fl(y,x)\imp sc(y)\right]\right] \\
  & \equiv \exists x\, \forall y\, \left[sc(x)\land ep(x) \land \left(\lnot fl(y,x) \lor sc(y)\right)\right]\\
3\spc& \forall x_2\, \left[sc(x_2)\imp \lnot hpe(x_2)\right]\\
  & \equiv \forall x_2\, \left[\lnot sc(x_2)\lor \lnot hpe(x_2)\right]\\
4\spc& \lnot \exists x_3\, \left[dn(x_3) \land sc(x_3)\right]\\
  & \equiv \forall x_3\, \left[\lnot dn(x_3) \lor \lnot sc(x_3)\right]
\end{align*}
}

\newcommand{\oneNODEtwo}{%
\begin{align*}
5\spc& \forall x\, \left[\left(\lnot ep(x_1) \lor hpe(x_1) \lor dn(h(x_1))\right) \land \left(\lnot ep(x_1) \lor hpe(x_1) \lor fl(h(x_1),x_1)\right)\right] & [\text{ext 1.}: \exists y/h(x_1)]\\
6\spc& \left(\left(\lnot ep(X_1) \lor hpe(X_1) \lor dn(h(X_1))\right) \land \left(\lnot ep(X_1) \lor hpe(X_1) \lor fl(h(X_1),X_1)\right)\right) & [\text{ext 5.}: \forall x_1/X_1]\\
7\spc& \lnot ep(X_1) \lor hpe(X_1) \lor dn(h(X_1)) & [\text{ext 6.}]\\
8\spc& \lnot ep(X_1) \lor hpe(X_1) \lor fl(h(X_1),X_1) & [\text{ext 6.}]\\
9\spc& \exists x\, \left[sc(x)\land ep(x) \land \left(\lnot fl(Y,x) \lor sc(Y)\right)\right] & [\text{ext 2.}: \forall y/Y] \\
10\spc& sc(a)\land ep(a) \land \left(\lnot fl(Y,a) \lor sc(Y)\right) & [\text{ext 9.}: \exists x/a] \\
11\spc& sc(a) & [\text{ext 10.}] \\
12\spc& ep(a) \land \left(\lnot fl(Y,a) \lor sc(Y)\right) & [\text{ext 10.}] \\
13\spc& ep(a) & [\text{ext 12.}] \\
14\spc& \lnot fl(Y,a) \lor sc(Y) & [\text{ext 12.}] \\
15\spc& \lnot sc(X_2)\lor \lnot hpe(X_2) & [\text{ext 3.}: \forall x_2/X_2] \\
16\spc& \lnot dn(X_3) \lor \lnot sc(X_3) & [\text{ext 4.}: \forall x_3/X_3] \\
\end{align*}
}

\newcommand{\nDsiete}{%
\begin{align*}
17\spc& \lnot ep(X_1) & [\text{br 7.}]\\
&\square~ [13.,17.]\\
& \sigma_1 = \left\{X_1\leftarrow a\right\}\\
\end{align*}
}

\newcommand{\nDocho}{%
\begin{align*}
18\spc& hpe(X_1) \lor dn(h(X_1)) & [\text{br 7.}]\\
\end{align*}
}

\newcommand{\nDnueve}{%
\begin{align*}
19\spc& \lnot sc(X_2) & [\text{br 15.}]\\
&\square~ [11.,19.]\\
& \sigma_2 = \left\{X_2\leftarrow a\right\}\\
\end{align*}
}

\newcommand{\nV}{%
\begin{align*}
20\spc& \lnot hpe(X_2) & [\text{br 15.}]\\
\end{align*}
}

\newcommand{\nVuno}{%
\begin{align*}
21\spc& hpe(X_1) & [\text{br 18.}]\\
&\square~ [20.,21.]\\
& \sigma_3 = \left\{X_1 = X_2\right\}\\
\end{align*}
}

\newcommand{\nVdos}{%
\begin{align*}
22\spc& dn(h(X_1)) & [\text{br 18.}]\\
\end{align*}
}

\newcommand{\nVtres}{%
\begin{align*}
23\spc& \lnot dn(X_3) & [\text{br 16.}]\\
&\square~ [22.,23.]\\
& \sigma_4 = \left\{X_3 = h(X_1)\right\}\\
\end{align*}
}

\newcommand{\nVcuatro}{%
\begin{align*}
24\spc& \lnot sc(X_3) & [\text{br 16.}]\\
\end{align*}
}

\newcommand{\nVcinco}{%
\begin{align*}
25\spc& \lnot fl(Y,a) & [\text{br 14.}]\\
\end{align*}
}

\newcommand{\nVseis}{%
\begin{align*}
26\spc&  sc(Y) & [\text{br 14.}]\\
&\square~ [24.,26.]\\
& \sigma_5 = \left\{Y = X_3\right\}\\
\end{align*}
}

\newcommand{\nVsiete}{%
\begin{align*}
27\spc&  \lnot ep(X_1) & [\text{br 8.}]\\
&\square~ [13.,27.]\\
& \sigma_6 = \left\{X_1\leftarrow a\right\}\\
\end{align*}
}

\newcommand{\nVocho}{%
\begin{align*}
28\spc& hpe(X_1) \lor fl(h(X_1),X_1) & [\text{br 8.}]\\
\end{align*}
}

\newcommand{\nVnueve}{%
\begin{align*}
29\spc& hpe(X_1) & [\text{br 28.}]\\
&\square~ [20.,29.]\\
&\sigma_7 = \left\{X_1=X_2\right\}\\
\end{align*}
}

\newcommand{\nT}{%
\begin{align*}
30\spc& fl(h(X_1),X_1) & [\text{br 28.}]\\
&\square~ [25.,30.]\\
&\sigma_8 = \left\{Y\leftarrow h(X_1), X_1 \leftarrow a\right\}\\
\end{align*}
}

\begin{document}
\pagenumbering{gobble}
{\bf TD 12. Exercice 2 :} Correction (Tableaux)

% Define block styles
\tikzset{
block/.style={
    %rectangle,
    %draw,
    text width=16em,
    text centered,
    %rounded corners
},
rectangle connector/.default=-2cm,
straight connector/.style={
    connector,
    to path=--(\tikztotarget) \tikztonodes
}
}

\begin{tikzpicture}[scale=0.8]
%\matrix (m)[matrix of nodes, column  sep=2cm,row  sep=8mm, align=center, nodes={rectangle,draw, anchor=center} ]{
\matrix (m)[matrix of nodes, column  sep=0.8cm,row  sep=5mm, align=center, nodes={anchor=center} ]{
|[block]|{\oneNODEone} & \\
|[block]|{\oneNODEtwo} & \\
|[block]|{\nDsiete} & |[block]| {\nDocho} \\
|[block]|{\nDnueve} & |[block]| {\nV} \\
|[block]|{\nVuno} & |[block]| {\nVdos} \\
%|[block]|{\nDseis} & |[block]| {\nDsiete}\\
};
\path [>=latex,-] (m-1-1) edge (m-2-1);
\path [>=latex,-] (m-2-1) edge (m-3-1);
\path [>=latex,-] (m-2-1) edge (m-3-2);
\path [>=latex,-] (m-3-2) edge (m-4-1);
\path [>=latex,-] (m-3-2) edge (m-4-2);
\path [>=latex,-] (m-4-2) edge (m-5-1);
\path [>=latex,-] (m-4-2) edge (m-5-2);
%\path [>=latex,-] (m-5-1) edge (m-6-1);
%\path [>=latex,-] (m-5-1) edge (m-6-2);
\end{tikzpicture}

\newpage 

\begin{tikzpicture}[scale=0.8]
%\matrix (m)[matrix of nodes, column  sep=2cm,row  sep=8mm, align=center, nodes={rectangle,draw, anchor=center} ]{
\matrix (m)[matrix of nodes, column  sep=0.8cm,row  sep=5mm, align=center, nodes={anchor=center} ]{
 & |[block]| {\nVdos} \\
|[block]|{\nVtres} & |[block]| {\nVcuatro} \\
|[block]|{\nVcinco} & |[block]| {\nVseis} \\
|[block]|{\nVsiete} & |[block]| {\nVocho}\\
|[block]|{\nVnueve} & |[block]| {\nT}\\
};
\path [>=latex,-] (m-1-2) edge (m-2-1);
\path [>=latex,-] (m-1-2) edge (m-2-2);
\path [>=latex,-] (m-2-2) edge (m-3-1);
\path [>=latex,-] (m-2-2) edge (m-3-2);
\path [>=latex,-] (m-3-1) edge (m-4-1);
\path [>=latex,-] (m-3-1) edge (m-4-2);
\path [>=latex,-] (m-4-2) edge (m-5-1);
\path [>=latex,-] (m-4-2) edge (m-5-2);
%\path [>=latex,-] (m-5-1) edge (m-6-1);
%\path [>=latex,-] (m-5-1) edge (m-6-2);
\end{tikzpicture}

$$\Sigma = \left\{X_1=X_2 \leftarrow a, X_3=Y \leftarrow h(a)\right\}$$

\end{document}