\documentclass[a4paper,10pt]{article}
\usepackage[utf8]{inputenc}
\usepackage{tikz}
\usetikzlibrary{matrix,shapes,arrows,positioning,chains}

\addtolength{\topmargin}{-.875in}
\addtolength{\textheight}{1.5in}
\addtolength{\textwidth}{1.7in}
\addtolength{\oddsidemargin}{-.875in}
\addtolength{\evensidemargin}{-.875in}

\usepackage{amsfonts}
\usepackage{amssymb}
\usepackage{amsmath}
\usepackage[section]{placeins}

\usepackage{turnstile}
\newcommand{\ded}[2]{#1\hspace{3pt}$\sdtstile{}{}$\hspace{3pt}#2}
\newcommand{\Ded}[1]{$\sdtstile{}{}$\hspace{3pt}#1}
\newcommand{\imp}{\Rightarrow}
\newcommand{\spc}{.\hspace{5pt}}

\newcommand{\oneNODEone}{%
\begin{align*}
1\spc& \forall x\, \left[ P(x)\imp\left(Q(x,c)\imp R\left(h(c)\right)\right)\right]\\
  & \equiv \forall x\, \left[\lnot P(x)\lor \lnot Q(x,c)\lor R\left(h(c)\right)\right]\\
2\spc& \forall u\,\left[P(u) \lor R(u)\right] \\
3\spc& \forall y\, \forall z\, \left[Q(z,y) \lor R(z)\right]\\
4\spc&\lnot \exists t\, R(t)\\
  & \equiv \forall t\, \lnot R(t)
\end{align*}
}

\newcommand{\oneNODEtwo}{%
\begin{align*}
5\spc& \lnot P(X)\lor \lnot Q(X,c)\lor R\left(h(c)\right) & [\text{ext 1.}: \forall x/X]\\
6\spc& P(U) \lor R(U) & [\text{ext 2.}: \forall u/U]\\
7,8\spc& Q(Z,Y) \lor R(Z) & [\text{ext 3.}: \forall z/Z, \forall y/Y]\\
9\spc& \lnot R(T) & [\text{ext 4.}: \forall t/T]
\end{align*}
}

\newcommand{\nDiez}{%
\begin{align*}
10\spc& P(U) & [\text{br 6.}]\\
\end{align*}
}

\newcommand{\nOnce}{%
\begin{align*}
11\spc& R(U) & [\text{br 6.}]\\
&\square~ [9.,11.]\\
& \sigma_1 = \left\{T=U\right\}\\
\end{align*}
}

\newcommand{\nDoce}{%
\begin{align*}
12\spc& \lnot Q(X,c) \lor R(h(c)) & [\text{br 5.}]\\
\end{align*}
}

\newcommand{\nTrece}{%
\begin{align*}
13\spc& \lnot P(X) & [\text{br 5.}]\\
&\square~ [10.,13.]\\
& \sigma_2 = \left\{X=U\right\}\\
\end{align*}
}

\newcommand{\nCatorce}{%
\begin{align*}
14\spc& \lnot Q(X,c) & [\text{br 12.}]\\
\end{align*}
}

\newcommand{\nQuince}{%
\begin{align*}
15\spc& R(h(c)) & [\text{br 12.}]\\
&\square~ [9.,15.]\\
& \sigma_3 = \left\{T\leftarrow h(c)\right\}\\
\end{align*}
}

\newcommand{\nDseis}{%
\begin{align*}
16\spc& Q(Z,Y) & [\text{br 8.}]\\
&\square~ [14.,16.]\\
& \sigma_4 = \left\{Z=X, Y\leftarrow c\right\}\\
\end{align*}
}

\newcommand{\nDsiete}{%
\begin{align*}
17\spc& R(Z) & [\text{br 8.}]\\
&\square~ [9.,17.]\\
& \sigma_5 = \left\{Z=T\right\}\\
\end{align*}
}

\begin{document}
\pagenumbering{gobble}
{\bf TD 12. Exercice 1 :} Correction (Tableaux)

% Define block styles
\tikzset{
block/.style={
    %rectangle,
    %draw,
    text width=16em,
    text centered,
    %rounded corners
},
rectangle connector/.default=-2cm,
straight connector/.style={
    connector,
    to path=--(\tikztotarget) \tikztonodes
}
}

\begin{tikzpicture}[scale=0.8]
%\matrix (m)[matrix of nodes, column  sep=2cm,row  sep=8mm, align=center, nodes={rectangle,draw, anchor=center} ]{
\matrix (m)[matrix of nodes, column  sep=0.8cm,row  sep=5mm, align=center, nodes={anchor=center} ]{
|[block]|{\oneNODEone} & \\
|[block]|{\oneNODEtwo} & \\
|[block]|{\nDiez} & |[block]| {\nOnce} \\
|[block]|{\nDoce} & |[block]| {\nTrece} \\
|[block]|{\nCatorce} & |[block]| {\nQuince} \\
|[block]|{\nDseis} & |[block]| {\nDsiete}\\
};
\path [>=latex,-] (m-1-1) edge (m-2-1);
\path [>=latex,-] (m-2-1) edge (m-3-1);
\path [>=latex,-] (m-2-1) edge (m-3-2);
\path [>=latex,-] (m-3-1) edge (m-4-1);
\path [>=latex,-] (m-3-1) edge (m-4-2);
\path [>=latex,-] (m-4-1) edge (m-5-1);
\path [>=latex,-] (m-4-1) edge (m-5-2);
\path [>=latex,-] (m-5-1) edge (m-6-1);
\path [>=latex,-] (m-5-1) edge (m-6-2);
\end{tikzpicture}

$$\Sigma = \left\{T=U=X=Z \leftarrow h(c), Y \leftarrow c\right\}$$

\end{document}