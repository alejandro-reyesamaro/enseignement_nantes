\documentclass[10pt,a4]{article}

\usepackage[latin1]{inputenc}
\usepackage{amsfonts}
\usepackage{amssymb}
\usepackage[ruled,vlined]{algorithm2e}
\usepackage[section]{placeins}

\newcommand{\comi}{\textquotedblleft}
\newcommand{\come}{\textquotedblright}

\renewcommand{\baselinestretch}{1.25}
%\pgfdeclareimage[width=3in, height=2in]{imagen1}{img1}

\parindent=0pt
%\parskip=8pt
\addtolength{\textheight}{0.6in}

\usepackage{turnstile}
\newcommand{\dednat}[2]{#1\hspace{10pt}$\sdtstile{N}{}$\hspace{10pt}#2}
\newcommand{\Dednat}[1]{$\sdtstile{N}{}$\hspace{10pt}#1}
\newcommand{\imp}{\Rightarrow}

\begin{document}
\begin{center}
\Large{\bf TD 8.} %\Large{\sc Forme PRENEXE et Skol�misation}
\end{center}

{\bf{{\large Exercice 1:}} } Exprimer en $L^1$ :
\begin{enumerate}
\item X et Y ont le m�me age;
\item X est plus �g� que Y;
\item \textit{marie} est la plus jeune fille de son groupe;
\item \textit{jacques} ne pourrait avoir que des amies blondes;
\item Toutes les amies de \textit{jacques} sont blondes;
\item Personne de Beatles n'a fait ses �tudes � Nantes;
\item[] \underline{Remarque} : utilisez le pr�dicat \textit{faire\_etudes}(Qui, O�, De, �).
\item \textit{``Momento mori''} : Si X est un homme, alors X est mortel;
\item \textit{Syllogisme d'Arystote} : 
\begin{enumerate}
\item \textit{A$_1$ ``Baroco''} : Si chaque P est M, mais s'il existe S qui n'est pas M, alors il existe S qui n'est pas P;
\item \textit{A$_2$ ``F�stimo''} : Si chaque P n'est pas M, mais il existe quelqu'un qui est S et M, alors il existe quelqu'un qui est S et qui n'est pas P.
\end{enumerate}
\item Soit $\phi$ une signature avec l'�galit� '=', exprimez :
\begin{enumerate}
\item \textit{f} est une fonction;
\item \textit{f} : P $\rightarrow$ Q.
\end{enumerate}
\end{enumerate}

{\bf{{\large Exercice 2:}} } Exprimez en $L^1$ et analysez :
\begin{enumerate}
\item Chaque math�maticien peut r�soudre ce probl�me � condition que quelqu'un puisse le r�soudre. Jean est un math�maticien et il ne peut pas r�soudre ce probl�me. Alors ce probl�me n'est pas d�cidable.
\item Tous les politiciens (politiques) sont com�diens. Il existe des com�diens qui sont hypocrites. Alors, il existe des politiciens qui sont hypocrites. 
\end{enumerate}

{\bf{{\large Exercice 3:}} } V�rifier que la proposition :
$$\exists x\, \left(p(x) \lor q(x)\right) \Leftrightarrow \exists x\, p(x) \lor \exists y\, q(y)$$
est \textbf{vrai} dans chaque domaine D � deux �l�ments.
\end{document}
